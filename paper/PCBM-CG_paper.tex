%JACS TEMPLATE STARTS HERE
\documentclass[journal=jacsat,manuscript=communication]{achemso}
\usepackage[version=3]{mhchem} % Formula subscripts using \ce{}
\usepackage{lineno,xcolor}
\newcommand*{\mycommand}[1]{\texttt{\emph{#1}}}

%\documentclass[aip,graphicx]{revtex4-1}
%\documentclass[aip,apl,amsmath,amssymb,linenumbers,reprint]{revtex4-1}
%\documentclass[aip,apl,amsmath,amssymb,linenumbers,preprint]{revtex4-1}
\usepackage{graphicx}
\usepackage[version=3]{mhchem} % Formula subscripts using \ce{}
\usepackage{subfig}
\usepackage{dcolumn}% Align table columns on decimal point

\usepackage{amsmath}% amsmath...
\usepackage{bm}% bold math

\usepackage{siunitx}% JMF addition - once a physicist, always a physicist
\usepackage{booktabs}% JMF addition - professional tables; no vertical lines

%\bibliographystyle{aipnum4-1}
%\draft % marks overfull lines with a black rule on the right

\title{PCBM-CG: A place for tired \LaTeX to rest}

\author{Jarvist M. Frost}
%\author{Federico Brivio}
%\author{Christopher H. Hendon}
\affiliation{Centre for Sustainable Chemical Technologies and Department of Chemistry, University of Bath, Claverton Down, Bath BA2 7AY, UK}

%\author{Mark van Schilfgaarde}
%\affiliation{Department of Physics, Kings College London, London WC2R 2LS, UK}

%\author{Aron Walsh}
%\email{a.walsh@bath.ac.uk}
%\affiliation{Centre for Sustainable Chemical Technologies and Department of Chemistry, University of Bath, Claverton Down, Bath BA2 7AY, UK}

\begin{document}

\begin{abstract}
Abstract. EPSRC gave us some money so we did our best to do great science, and these are our conclusions. 
\end{abstract}

%\pacs{88.40.-j, 71.20.Nr, 72.40.+w, 61.66.Fn}
% 71.20.Nr 	Semiconductor compounds 
% 72.40.+w 	Photoconduction and photovoltaic effects
% 61.66.Fn 	Inorganic compounds 
% 88.40.jn 	Thin film Cu-based I-III-VI2 solar cells
% 88.40.-j 	Solar energy

%\maketitle 

\textbf{Lambdas}

\begin{table}
\centering
\begin{tabular}{lSSS}
\toprule
Isomer & $\lambda_{neut}$ & $\lambda_{ion}$ & $\lambda_{tot}$ \\
\midrule
mono & 77.91 & 77.49 & 155.40 \\
\midrule
bis-C1 & 111.52 & 182.64 & 294.16 \\
bis-C2 & 108.54 & 158.89 & 267.43 \\
bis-C3 & 81.38 & 83.31 & 164.69 \\
bis-E1 & 88.82 & 89.49 & 178.31 \\
bis-T1 & 138.30 & 151.32 & 289.62 \\
bis-T2 & 80.30 & 80.93 & 161.23 \\
bis-T3 & 125.77 & 166.20 & 291.97 \\
bis-T4 & 87.66 & 95.56 & 183.22 \\
\midrule
tris-E,E,E & 108.42 & 105.41 & 213.84 \\
tris-E,E,T1(1) & 99.51 & 100.82 & 200.33 \\
tris-E,E,T1(2) & 94.62 & 98.86 & 193.49 \\
tris-E,T3,T2 & 93.97 & 92.93 & 186.90 \\
tris-E,T4,T2 & 98.54 & 106.46 & 205.00 \\
tris-E,T4,T3 & 100.51 & 100.06 & 200.56 \\
tris-T3,T3,T3 & 137.97 & 173.63 & 311.60 \\
tris-T4,T3,T3 & 200.30 & 226.34 & 426.64 \\
tris-T4,T4,T2 & 149.22 & 148.26 & 297.48 \\
%tris- & {DATA} & {HERE} & {PLEASE} \\
tris-T4,T4,T4 & 136.01 & 166.56 & 302.57 \\
\bottomrule
\end{tabular}
\caption{\label{tab:Lambda}
Inner sphere reorganisation energies of Mono, Bis and Tris PC$-{60}$BM fullerenes. All units meV.}
\end{table}

\begin{table}
\centering
\begin{tabular}{lSSS}
\toprule
Isomer & $\lambda_{neut}$ & $\lambda_{ion}$ & $\lambda_{tot}$ \\
\midrule
mono & 68.80 & 68.83 & 137.63 \\
\midrule
bis-c1 & 70.42 & 70.57 & 140.99 \\
bis-c2 & 70.45 & 73.48 & 143.93 \\
bis-c3 & 78.69 & 72.39 & 151.08 \\
bis-e1 & 69.79 & 74.79 & 144.58 \\
bis-t1 & 69.37 & 69.01 & 138.38 \\
bis-t2 & 78.85 & 79.53 & 158.38 \\
bis-t3 & 75.42 & 72.98 & 148.40 \\
bis-t4 & 70.03 & 68.63 & 138.66 \\
\midrule
tris-EEE & 105.49 & 78.63 & 184.12 \\
tris-EET1-1 & 76.11 & 76.12 & 152.22 \\
tris-EET1-2 & 72.66 & 72.96 & 145.62 \\
tris-ET3T2 & 73.33 & 75.39 & 148.72 \\
tris-ET4T2 & 75.29 & 72.51 & 147.80 \\
tris-ET4T3 & 76.80 & 73.70 & 150.50 \\
tris-T3T3T3 & 82.08 & 75.01 & 157.09 \\
tris-T4T3T3 & 77.78 & 78.22 & 155.99 \\
tris-T4T4T2 & 74.82 & 74.58 & 149.40 \\
tris-T4T4T4 & 0.00 & 0.00 & 0.00 \\
\midrule
c70-mono & 92.31 & 87.75 & 180.07 \\
c70-bis-4158 & 98.25 & 96.06 & 194.31 \\
c70-bis-5657 & 87.27 & 89.48 & 176.75 \\
c70-bis-6768 & 96.07 & 96.10 & 192.17 \\
\bottomrule
\end{tabular}
\caption{\label{tab:Lambda}
Inner sphere reorganisation energies of Mono, Bis and Tris Methano fullerenes. All units meV.}
\end{table}




\textbf{Some mobs}


\begin{table}
\centering
\sisetup{scientific-notation=fixed,fixed-exponent=-3,round-mode=figures,round-precision=3}
\newcolumntype{H}{>{\lrbox0}c<{\endlrbox}@{}} % Hide some data :^)
\begin{tabular}{lSHSHSH}
\toprule
$\sigma$ & 0.0 & 0.021 & 0.056 & 0.072 & 0.121 & 0.250 \\
\midrule
 M  & 0.004402762    &  0.004071634      &  0.002722337      &  0.002132572      &  0.0008374095     &  4.014195e-05     \\
 B  & 0.002272817    &  0.002067089      &  0.001300517      &  0.0009679719     &  0.0003292884     &  1.183156e-05     \\
 B-E1  & 0.00188074      &  0.00172335   &  0.001089123      &  0.0008162235     &  0.0002771943     &  1.031878e-05     \\
 T  & 0.001201081    &  0.001014771      &  0.0005888677     &  0.000400768      &  0.0001261058     &  3.709921e-06     \\
 T-EEE  & 0.0006229742   &  0.0006152113     &  0.0004291373     &  0.0003060389     &  8.536267e-05     &  2.495028e-06 \\
\bottomrule
\end{tabular}
\caption{\label{tab:mobs}
Simulated mobility by Time of Flight (using the \textsc{ToFET} code), with varying energetic disorder. Units are \si{cm^2/Vs}}
\end{table}

\begin{acknowledgement}
We acknowledge membership of the UK's HPC Materials Chemistry Consortium, which is funded by EPSRC grant EP/F067496. 
J.M.F. is funded by EPSRC Grant EP/K016288/1.
%J.M.F. and K.T.B. are funded by EPSRC Grants EP/K016288/1 and EP/J017361/1, respectively.
%F.B. is funded through the EU DESTINY Network (Grant 316494).
%C.H.H. is funded by ERC (Grant 277757). 
%A.W. acknowledges support from the Royal Society and ERC (Grant 277757). 
We are grateful for the lyrical encouragment of Salt N Pepa. 
\end{acknowledgement}

\begin{suppinfo}
    The data set and analysis codes, \textsc{TrendyName}, are available as a source code repository on GitHub\cite{GitHub}.
\end{suppinfo}

\bibliography{library}

\end{document}
